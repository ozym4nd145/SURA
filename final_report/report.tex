\documentclass[12pt]{article}
\usepackage[utf8]{inputenc}
\usepackage{graphicx}
\usepackage{titlepic}
\usepackage{caption}
\usepackage{subcaption}
\usepackage[a4paper, total={6in, 8in}]{geometry}

% \documentclass{beamer}
\usepackage{amsmath}

\newcommand{\namesigdate}[2][5cm]{%
  \begin{tabular}{@{}p{#1}@{}}
    #2 \\[0.4\normalbaselineskip] \hrule \\[0pt]
    {\small } \\[2\normalbaselineskip] 
  \end{tabular}
}

\title{\vspace*{\fill} \textbf{Video Description using Deep Learning}
	  \\ {\large \textbf{Summer Undergraduate Research Award}}
	  \\  \vspace{3mm} \includegraphics[width=5cm]{logo.jpg}}

\author{
	\textbf{Suyash Agrawal}\\ 
	2015CS10262\\
	Computer Science\\
	CGPA: 9.87 \\
	Mob: 9717060183\\
	cs1150262@iitd.ac.in
	\and
	\textbf{Madhur Singhal}\\ 
	2015CS10235\\
	Computer Science\\
	CGPA: 8.66\\
	Mob: 9540972599\\
	cs1150235@iitd.ac.in
}
\date{\textbf{Supervisor:-} \\ \textbf{Subhashis Banerjee} \\ Professor \\ Department of CSE \\ suban@cse.iitd.ac.in\\ IIT Delhi\\
\vspace*{\fill}}




\begin{document}
	\maketitle

\begin{center}
\noindent\rule{3.2cm}{0.4pt} 
\end{center}

\begin{flushright}
\noindent\rule{3.2cm}{0.4pt} 
\\ \textbf{Prof. S. Arun Kumar}
\\ Head of Department
\\ Department of CSE
\\ sak@cse.iitd.ernet.in
\end{flushright}


	\newpage

	\section{Introduction}
		\textit{\textbf{Video Description}} is the process of discovering knowledge, structures, patterns and events of interest in the video data and describing them in natural language. Video Description is an open problem in computer vision and currently the only source of video description is manual labour. 
		\newline
		Video Description has wide variety of applications. It can help visually impaired people ``see'' the world by describing the scene around them. It has also use in automated surveillance by analyzing the videos in real time and reporting malicious or unusual activities. It can also be used to efficiently index large video databases based upon their content for ease of accessibility.

		\begin{figure}[ht!]
		\centering
		\includegraphics[width=1.0\textwidth]{description.png}
		\caption{Sample Video Description\label{fig0}}
		\end{figure}		


	\section{Past Work}

		Much of the prior work in this field is focused on generating natural language descriptions from images. Recently the use of an end to end deep neural network architecture which takes an image as input and generates an english description has  been shown to give excellent results. We extended this approach to a Video Description network which takes in a sequence of frames as input and generates an english description of the actions happening in the video. For this we use an encoder-decoder framework widely used in Machine Translation, which maps a variable length input to a variable length output. 

	\section{Data}
The data used is a crucial factor in the effectiveness of a complex neural network like ours. We primarily used the following two datasets for training and validation purposes.
	\begin{enumerate}
		\item
			\textbf{Microsoft Research - Video to Text (MSR-VTT)}: The dataset contains 41.2 hours and 200K clip-sentence pairs in total, covering the most comprehensive categories and diverse visual content, and representing the largest dataset in terms of sentence and vocabulary.
		\item
			\textbf{Montreal Video Annotation Dataset (M-VAD)}: The M-VAD movie description corpus is another recent collection of about 49,000 short video clips from 92 movies. It is similar to MPII-MD, but only contains AD data and only provides automatic alignment.
		\end{enumerate}
		Both of these datasets consist of short Youtube video clips (average length near 20 seconds) and english sentences (average length near 12 words) describing the clips. Most of the video clips also have a synchronized audio clip which we utilize in some of our experiments. 
		Empirically we found out that these datasets were big enough for our network to learn proper English language sentence structure and grammar automatically, and the sentences we return are almost always gramatically correct. The datasets also identify a wide host of objects and actions which make the network applicable in general situations.\\
		The main problem with these datasets is that they were not properly sampled and thus there was an inherent bias towards the types of videos seen on Youtube as opposed to those encountered in the real world. For example, the datasets had a large amount of videos of video games which often involve rapidly changing frames and our network began to associate things like car crashes to video game clips. 
\section{Methods}
\subsection{Overall Approach}
	Our approach to video description was motivated by the recent successful approach of using Encoder-Decoder system in Image Description.
	Our pipeline consisted of a CNN layer that extracted feateures from the video, then one LSTM network which encoded these features and finally
	another LSTM network for decoding into natural language description.

\subsection{Video Feature Extraction}

	We used Convolutional neural networks for extracting features from video frames. In recent times CNNs have shown amazing results in object
	detection in images and thus they form a natural choice in extracting features from images. Also, since videos are a sequences of images, we
	downsample video frames at a fixed frame rate and then individually extract features from each video frame and concantenate them to get 
	the feature vector of the video.\\
	We used Inception V4 net as our choice of CNN for video feature extraction as it has shown the best results in ImageNet challenge and is thus
	more likely to give better feature extraction from image frames. We also experimented with audio features in the video. Specifially we looked
	at the MFCC features of the audio and we processed these features along with image features in our encoder network to get better representation
	of the video. But we later abandoned this approach as it did not show any significant improvement in the results and also added audio dependency
	to our network which would have made this un-usable in the case of video surveillance and other areas where audio is not available.
	
\subsection{Video Feature Encoding}
	We used a LSTM based network for encoding features of video frame into a single feature vector. This approach is largely inspired by recent
	breakthrough of similar approach in machine translation which also used LSTMs for encoding decoding language token. Here we used the feature
	vector of video frames as inputs to LSTM units and the final hidden state of the LSTM network as the video encoding.\\ 
	We experimented with various types of encoding networks like:
	\begin{itemize}
		\item Single layer LSTM network
		\item Multi layered LSTM network (2-5 layers)
		\item Residual Network Multi Layered LSTM network
		\item Encoder network with Attention Model
	\end{itemize}
	Apart from the above, we also experimented with dropout in between layers and also experimented with audio features concatenated with
	video features.

\section{Approach to the project}
		The basic approach is to train our model using videos that already have a description, so that our networks ``learns'' to describe videos.\\
		In order to achieve this we have broken down our problem in several parts. First, we will be implementing image captioning in order to be able to encode individual video frames in fixed-length vector representation.
		Then, we will be using our constructed image captioning network's CNN (image encoder) to encode video frames sampled at a fixed interval.
		We will then use Recurrent Neural Networks to translate this representation of video into natural language domain. This translation will be achieved by using a set of encoder and decoder RNNs. Since we are working with variable length input and output, we will specifically be using LSTMs (Long Short Term Memory Networks) for encoding and decoding purposes, as they have been proven to be excellent in machine translation and generalize very well on long data input.
		\begin{enumerate}
			\item Image Captioning
			\begin{enumerate}
				\item
					Construct a convolutional neural network (VGG/Inception V3) with pre-trained weights for object classification.
				\item
					Make an RNN (LSTM) network\cite{showandtell} that takes a encoded image (using CNN) as input and then translates the image representation into natural language text.
				\item
					Train the model on datasets like MS COCO which consists of large number of captioned images.
			\end{enumerate}
			\item Video Representation
			\begin{enumerate}
				\item
					We will first sample video at a fixed rate and convert each individual frame to a fixed length vector representation using the network trained in image captioning part.
				\item
					We will then try out different approaches of video representation like:
					\begin{itemize}
						\item
							Mean Pooling over all video frame encodings obtained to get a fixed vector representation of video\cite{proposal}.
						\item
							Using RNNs to encode this variable length video representation\cite{s2vt}.
						\item
							Using 3D CNNs to directly encode videos and use simple RNNs for description generation\cite{temporal}.
					\end{itemize}					
			\end{enumerate}
			\item Natural Language Conversion
			\begin{enumerate}
				\item
					Based on how we choose to encode our video, we will have to select appropriate architecture of LSTM to be used to decode the representation
				\begin{figure}[ht!]
				\centering
					\includegraphics[width=10cm]{s2vt.png}
					\caption{Video description model with 2 LSTM levels\label{fig1}}
				\end{figure}

				\item
					One popular choice\cite{s2vt}(Figure~\ref{fig1}) that we will try out first will be to use a two level LSTM model that will do a sequence to sequence mapping from variable length video representation to variable length natural language sentence.
				\item
					Then we will train our model on the training data we have obtained and plot the learning curves.
				\item
					We will also have to check for over-fitting and under-fitting during our training process and finetune our hyper parameters according to it.
			\end{enumerate}
			\item Further Possibilities
			\begin{enumerate}
				\item We will look for techniques of data augmentation and transfer learning\cite{proposal} to compensate for the limited amount of training data for video description.
				\item We will also look into some recent techniques of using optical flow for attention modelling\cite{s2vt} which has shown in some cases to improve the results of action recognition.
				\item We will also consider using more efficient vocabulary representation as compared to one hot encoding because vocabulary in this case would be very large.
				\item Try to make forward propagation fast and more memory efficient.
				\item We will also try to develop an end user application that will speak out description of a video that a person records.
				\item A future extension of this project is to design a system which can answer questions based upon a video, similar to \cite{visualqa}.
			\end{enumerate}
		\end{enumerate}

	

	\section{Uses and applications}
			\begin{itemize}
				\item
					Assisting visually impaired people to get description of their surroundings, thus enabling them to ``see''.
				\item
					Very useful for automated surveillance and theft detection by being able to analyze large amounts of data which is unfeasible to be done by humans.
				\item
					Allowing content based video retrieval by describing the contents of video in textual format which is indexable by web crawlers.
				\item
					This can also be used to detect catastrophic events through security cameras like fire breakout, murder etc.
				\item
					This project can also be applied in helping robotic vision as this project basically allows one to understand what is happening in the video and thus robots will be able to get a ``true'' sense of their surroundings.
			\end{itemize}



	\section{Background} 
			\subsection{Deep Learning}
				Deep Learning is a branch of machine learning in which multiple parameter based models are used in series. In a deep network, there are many layers between the input and output, allowing the algorithm to be executed in multiple processing steps, composed of \textbf{multiple linear and non-linear transformations}. At each layer, the signal is transformed by a processing unit, like an artificial neuron, whose parameters are \textbf{`learned'} through training. Deep Learning has been shown to excel in tasks where the goal is to find \textbf{intuitive} patterns in the data.\cite{deep} In particular, in the field of Computer Vision, deep networks are increasingly used to extract \textbf{feature descriptions and inter-relationships between features} from images.\cite{cs231n}
				\begin{figure}[ht!]
					\includegraphics[width=14cm]{blog_deeplearning3.jpg}
					\caption{Illustration of Deep Learning as applied to Vision\label{fig2}}
				\end{figure}	

			\subsection{Convolutional Neural Networks}
			Convolutional Neural Networks (CNN, or ConvNet) are a type of feed-forward artificial neural network in which the connectivity pattern between the neurons is inspired by the organization of the animal visual cortex. Individual cortical neurons respond to stimuli in a restricted region of space known as the \textbf{receptive field}. The receptive fields of different neurons partially overlap such that they tile the visual field. The response of an individual neuron to stimuli within its receptive field can be approximated mathematically by a \textbf{convolution operation}. A Convolutional Neural Network consists of the following layers.\cite{showandtell}
								
				\begin{figure}[ht!]
					\includegraphics[width=1.0\textwidth]{conv.png}
					\caption{A Typical Convolutional Neural Network\label{fig4}}
				\end{figure}
			%skipping Relu layer since its not in picture assume it to be in conv layer
				\subsubsection{Convolutional Layer}
					The convolution layer is the core building block of a CNN. The layer's parameters consist of a set of \textbf{learnable filters} (or kernels), which have a small receptive field, but extend through the full depth of the input volume. During the forward pass, each filter is convolved across the width and height of the input volume, computing the dot product between the entries of the filter and the input and producing a $2$-dimensional activation map of that filter.\cite{cs231n} As a result, the network learns filters that activate when it detects some specific type of feature at some spatial position in the input.

				\subsubsection{Max Pooling Layer}
					It is common to periodically insert a Pooling layer in-between successive Conv layers in a ConvNet architecture. Its function is to \textbf{progressively reduce the spatial size} of the representation to reduce the amount of parameters and computation in the network, and hence to also control over-fitting. The Pooling Layer operates independently on every depth slice of the input and resizes it spatially, using the max operation.\cite{cs231n}
				\subsubsection{Fully-Connected Layer}
					Finally, after several convolutional and max pooling layers, the high-level reasoning in the neural network is done via fully connected layers. Neurons in a fully connected layer have \textbf{full connections to all activations} in the previous layer, as seen in regular Neural Networks. Their activations can hence be computed with a matrix multiplication followed by a bias offset.\cite{fullyconnected} Thus output of the fully connected layer is a vector with elements representing the `probability' (not in a strictly statistical sense) of the image containing specific objects or actions.

			\subsection{Long Short Term Memory Networks}
			    \begin{figure}[ht!]
			    	\centering
					\includegraphics[scale=0.266]{LSTM_unit.png}
					\caption{A single LSTM unit\label{fig6}}
				\end{figure}
				Long Short Term Memory Networks are a type of Recurrent Neural Networks. These networks are based upon recursion, so that variable length inputs can be handled easily and sequential information can be processed with better results. LSTM's are specifically used for making RNNs learn long term patterns since traditional RNNs tend to \textbf{favour short term temporal dynamics}. It can be difficult to train traditional RNNs to learn long-term dynamics, likely due in part to the \textbf{vanishing and exploding gradients problem} that can result from propagating the gradients down through the many layers of the recurrent network, each corresponding to a particular time step\cite{ltms}. LSTMs provide a solution by incorporating memory units that explicitly allow the network to learn when to ``forget'' previous hidden states and when to update hidden states given new information\cite{lstmexecute}.\\
%				Below we list the main equations governing the behaviour of the LSTM networks.\\
%				\begin{align}
%					f_{t} &= \sigma_{g}(W_{f}x_{t} + U_{f}h_{t-1} + b_{f})\\	
%					i_{t} &= \sigma_{g}(W_{i}x_{t} + U_{i}h_{t-1} + b_{i})\\
%					o_{t} &= \sigma_{g}(W_{o}x_{t} + U_{o}h_{t-1} + b_{o})\\
%					c_{t} &= f_{t} \circ c_{t-1} + i_{t} \circ \sigma_{c}(W_{c}x_{t} + U_{c}h_{t-1} + b_{c})\\
%					h_{t} &= o_{t} \circ \sigma_{h}(c_{t})
%				\end{align}
%				The symbol meanings are:\\
%	$\mathbf{x_{t}}$: Input vector\\
%    $\mathbf{h_{t}}$: Output vector\\
%    $\mathbf{c_{t}}$: Cell state vector\\
%    $\mathbf{W}$, $\mathbf{U}$ and $\mathbf{b}$: Parameter matrices and vector\\
%    $\mathbf{f_{t}}$: Forget gate vector. Weight of remembering old information.\\
%    $\mathbf{i_{t}}$: Input gate vector. Weight of acquiring new information.\\
%    $\mathbf{o_{t}}$: Output gate vector. Output candidate\\
%    $\mathbf{\sigma_{g}}$, $\mathbf{\sigma_{c}}$ and $\mathbf{\sigma_{h}}$: Activation functions\\
		
			\subsection{Training Deep Neural Networks}
					\begin{figure}[ht!]
					\includegraphics[width=14cm]{training_inference1.png}
					\caption{Training and Inference Processes\label{fig5}}
				\end{figure}
				A Deep Neural Network is at it's core a parameter based function. All of these parameters are  \textbf{trained} automatically from inputs and expected output tuples (training data). The training process revolves around minimizing a particular cost function using methods like \textbf{Stochastic gradient descent}. The input is given to the network in a feed forward fashion and the parameters are modified from the last layer to the first \textbf{(Backpropagation)}. Neural Networks, by design, require huge amounts of training data and take a large time to get trained. For some perspective, most current state of the art image classifiers have $> 100$ million parameters and are trained on more than 1.2 million images. 


			\subsection{Finetuning}
				Fine-tuning a network is a procedure based on the concept of
				\textbf{transfer learning}. We start training a CNN to learn features for a broad domain with a
				classification function targeted at minimizing error in that domain. Then, we
				replace the classification function and \textbf{optimize the network} again to minimize
				error in another, more specific domain. Under this setting, we are transferring
				the features and the parameters of the network from the broad domain to the
				special one.\cite{fineplant} In our project we will need to use the pre-trained image classification models 
				to actually decode individual frames of the video, thus we are planning to \textbf{fine-tune those models
				with respect to the output of our LSTMs}.



	\begin{thebibliography}{1}
	
	  \bibitem{proposal} Subhashini Venugopalan {\em Natural Language Video Description using Deep Recurrent Neural Networks}, 2015.
	
	  \bibitem{s2vt}  Venugopalan, Subhashini and Rohrbach, Marcus and Donahue, Jeff
                    and Mooney, Raymond and Darrell, Trevor and Saenko, Kate {\em Proceedings of the IEEE International Conference on Computer Vision (ICCV)}, 2015
	
	  \bibitem{showandtell} Oriol Vinyals and
               Alexander Toshev and
               Samy Bengio and
               Dumitru Erhan {\em Show and Tell: {A} Neural Image Caption Generator} 2014.

        \bibitem{lstmexecute} Wojciech Zaremba, Ilya Sutskever {\em Learning to Execute} ICLR 2015
         

         \bibitem{temporal}  Li Yao, Atousa Torabi, Kyunghyun Cho, Nicolas Ballas, Christopher Pal, Hugo Larochelle, Aaron Courville {\em Describing Videos by Exploiting Temporal Structure} ICCV 2015
        

         \bibitem{visualqa} Aishwarya Agrawal, Jiasen Lu , Stanislaw Antol,
		Margaret Mitchell, C. Lawrence Zitnick, Dhruv Batra, Devi Parikh {\em VQA: Visual Question Answering} 2016
        

         \bibitem{ltms} Jeff Donahue, Lisa Anne Hendricks, Marcus Rohrbach, Subhashini Venugopalan, Sergio Guadarrama, Kate Saenko, Trevor Darrell {\em Long-term Recurrent Convolutional Networks for Visual Recognition and Description} 2016

         \bibitem{deep} Bengio, Yoshua; LeCun, Yann; Hinton, Geoffrey {\em Deep Learning} 2015

         \bibitem{fineplant} Angie K. Reyes, Juan C. Caicedo and Jorge E. Camargo {\em Fine-tuning Deep Convolutional Networks for
		Plant Recognition} LifeCLEF 2015

         \bibitem{cs231n} Andrej Karpathy {\em 
			CS231n Convolutional Neural Networks for Visual Recognition
		} http://cs231n.github.io/convolutional-networks/

		\bibitem{fullyconnected} Santanu Chaudhury, Anupama Mallik, Hiranmay Ghosh {\em Multimedia Ontology: Representation and Applications
		} 

	\end{thebibliography}
\end{document}
